\section{Introduction}
手写体字符识别是一个几乎解决了许多主流语言的问题,得益于最近深入学习模型的进展。尽管如此,对于许多其他语言来说,手写体数字识别仍然是一个具有挑战性的问题,因为缺乏足够大的标记数据集来训练深度学习模型。而传统的模型作为线性分类器,K-最近邻分类器、非线性分类器和支持向量机可以被用于这项任务的,但是它们不能实现深度学习模式所提供的接近人类水平的性能。卷积神经网络(CNN)由于具有编码深层特征的能力而取得了最先进的成果。尽管CNN善于理解图像中的低层和高层特性,但这样做,它们在池化层时会丢失有价值的信息。CNN需要大量的训练样本(通常为数千或数万级),以成功训练和分类图像。

TextCaps : Handwritten Character Recognition with Very Small Datasets \cite{jayasundara2019textcaps}是2019年发表在WACV的一篇文章。在这篇文章中,作者提出了一种利用胶囊网络(Capnets)\cite{sabour2017dynamic}解决标记数据集大小小的问题的技术。仅通过利用他们操纵实例化参数\cite{hinton2011transforming}来增强数据的能力。Capnets可以学习图像的属性-在这种情况下是一个字符。这让他们在学习用较少的标记数据来识别字符方面很有用。我们是基于Sabour等人提出的CapsNet架构。\cite{sabour2017dynamic}它包括一个胶囊网络和一个全连接解码网络。我们在对胶囊网络做小改动的同时,用反卷积网络代替了解码器网络。通过在实例化中添加受控制的噪声量,我们对实体进行转换,以描述实际发生的变化。这就产生了一种新的数据生成技术,比仿射变换增强数据更真实。由于重建精度在许多情况下也很重要,我们提出了一种结合损失函数的经验合适的策略。 大大改善了重建效果。我们的系统达到了与世界先进水平相媲美的效果,每类只有200个样本,同时也取得了更好的效果。

本文主要贡献:

1. 在三个数据集上进行了训练并超过了最好的结果,三个数据集分别是EMNIST-letters, EMNIST-balanced 和 EMNIST-digits

2. 在非字符数据集上进行了鲁棒性分析,该数据集是 Fashion-MNIST

3. 每类只用200个样本训练胶囊网络,同时测试样本与之前相当。获得了同样的效果。

4. 分析不同损失函数的效果

注:数据集简介:EMNIST-letters字母数据集共26类、EMNIST-balanced 字母数字数据集共47类、EMNIST-digits数字数据集共10类、Fashion-MNIST时尚单品图像数据集共10类
