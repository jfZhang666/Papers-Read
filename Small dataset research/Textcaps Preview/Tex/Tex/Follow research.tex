\section{Follow Research}
\subsection{DeepCaps: Going Deeper with Capsule Networks}
本文摘要翻译\cite{rajasegaran2019deepcaps}
胶囊网络在深度学习中是一个很有前途的概念,目前他在几个关键的数据集上的的表现低于标准结果,其真正的潜力还没有得到充分的发挥。深度卷积神经网络的成功给我们带来了启示,我们介绍DeepCaps,顾名思义它是一个深度胶囊网络。它基于最新的三维卷积算法。 使用DeepCaps,我们在CIFAR10,SVHN和Fashion MNIST的胶囊网络领域中超越了最新技术成果,同时使参数数量减少了68%。

此外,我们提出了一个与类无关的解码器网络,该网络加强了重建损失作为正则化项的使用。 这导致了解码器的有趣特性,它使我们能够识别和控制由实例化参数表示的图像的物理属性。

本文核心:提出了DeepCaps网络,该网络对于数据集CIFAR10, SVHN和 Fashion MNIST的结果超过了以往的胶囊网络识别结果。

\subsection{TreeCaps: Tree-Structured Capsule Networks for Program Source Code Processing}
本文摘要翻译\cite{jayasundara2019treecaps}:程序理解是软件开发和维护过程中的一项基本任务。软件开发人员通常需要先了解大量现有代码,然后才能开发新功能或修复现有程序中的错误。能够自动处理编程语言代码并准确提供代码功能摘要可以极大地帮助开发人员减少花在代码导航和理解上的时间,从而提高生产率。与自然语言文章不同,编程语言中的源代码通常遵循严格的语法结构,并且通过复杂的控制流和数据流,彼此远离的代码元素之间可能存在依赖关系。对基于树的卷积神经网络(TBCNN)和门控图神经网络(GGNN)的现有研究无法准确地捕获代码元素之间的基本语义依赖性。在本文中,我们提出了新颖的基于树的胶囊网络(TreeCaps)和以自动方式处理程序代码的相关技术,该方法对代码的语法结构进行编码并更准确地捕获代码依赖性。基于对用不同编程语言编写的程序的评估,我们表明,在对许多程序的功能进行分类时,基于TreeCaps的方法可以胜过其他方法。

本文核心:基于胶囊网络的程序理解以及处理